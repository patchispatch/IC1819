\documentclass[11pt,a4paper]{article}
\usepackage[utf8]{inputenc}
\usepackage[spanish]{babel}	%Idioma
\usepackage{amsmath}
\usepackage{amsfonts}
\usepackage{amssymb}
\usepackage{graphicx} %Añadir imágenes
\usepackage{geometry}	%Ajustar márgenes
\usepackage[export]{adjustbox}[2011/08/13]
\usepackage{float}
\restylefloat{table}
\usepackage{hyperref}
\usepackage{titling}
%\usepackage{minted}
\usepackage[font=small,labelfont=bf]{caption} 

%Opciones de encabezado y pie de página:
\usepackage{fancyhdr}
\pagestyle{fancy}
\lhead{}
%\rhead{}
\lfoot{Ingeniería del Conocimiento}
\cfoot{}
\rfoot{\thepage}
\renewcommand{\headrulewidth}{0.4pt}
\renewcommand{\footrulewidth}{0.4pt}
%Opciones de fuente:
\usepackage[utf8]{inputenc}
\usepackage[default]{sourcesanspro}
\usepackage{sourcecodepro}
\usepackage[T1]{fontenc}


\setlength{\parindent}{0pt}
\setlength{\headheight}{15pt}
\setlength{\voffset}{10mm}

% Custom colors
\usepackage{color}
\definecolor{deepblue}{rgb}{0,0,0.5}
\definecolor{deepred}{rgb}{0.6,0,0}
\definecolor{deepgreen}{rgb}{0,0.5,0}

\usepackage{listings}

% Evitar guiones al final de línea.
\tolerance=1
\emergencystretch=\maxdimen
\hyphenpenalty=10000
\hbadness=10000

\pretitle{%
  \centering
  \LARGE
  \includegraphics[scale=0.5]{../../../../logo.png}\\[\bigskipamount]
}
\posttitle{\begin{center} \end{center}}

\author{Juan Ocaña Valenzuela}
\title{\textbf{Ingeniería del Conocimiento} \\ 
 Sistema Experto para clasificar SPAM: }

%%%%%%%%%%%%%%%%%%%%%%%%%%%%%%%%%%%%%%%%%%%%%%%%%%%%%%%%%%%%%%%%%%%%%%%%%%%%%%%
%% EL DOCUMENTO EMPIEZA AQUÍ
%%%%%%%%%%%%%%%%%%%%%%%%%%%%%%%%%%%%%%%%%%%%%%%%%%%%%%%%%%%%%%%%%%%%%%%%%%%%%%%

\begin{document}

\thispagestyle{empty}

\maketitle

\begin{center}
%\url{https://github.com/patchispatch}

Versión: 1.0
\end{center}

\newpage

\tableofcontents

\newpage

\section{Conocimiento propio para clasificar un email como SPAM}

Un correo SPAM es aquel no deseado por el receptor, ya sea porque lo ha enviado una 
compañía con intenciones publicitarias, un anónimo que los servidores han marcado como 
"peligroso" u otros que el propio usuario haya restringido. Un servidor de correo debe ser
capaz de detectar este correo y almacenarlo en la respectiva carpeta SPAM del usuario.

Un correo SPAM puede ser considerado así si cumple estas condiciones:

\begin{itemize}
\item El correo proviene de una fuente marcada por el servidor como spam.
\item El correo proviene de una fuente marcada por el usuario como spam.
\item El servidor ha recibido ese mismo correo muchas veces, para diferentes usuarios, en 
una ventana de tiempo acotada.
\item El correo contiene un enlace que el servidor ha marcado como malicioso.
\item El correo contiene archivos adjuntos que son potencial malware.
\end{itemize}

\section{Mayor dificultad para que el sistema experto categorice SPAM}
Considero que la mayor dificultad sería no vulnerar la privacidad de los usuarios para categorizar spam,
ya que las medidas anteriormente mencionadas requieren acceder al contenido del mensaje.

\medskip

Suponiendo que no existiese este problema, el Sistema Experto tendría que implementar
una lista negra de fuentes y enlaces maliciosos. El Sistema Experto debería, además, estimar 
en qué medida pasa un correo sus filtros, y recomendar si marcarlo como spam o no si supera un determinado
umbral.

\end{document}